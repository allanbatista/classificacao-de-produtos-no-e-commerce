\chapter{Resultados}

Os resultados desse trabalho, foram todos feitos com dados reais e representação a qualidade do modelo neste momento.

\section{MÉTODOS DE AVALIAÇÃO}

A comparação da performance dos modelos foram feitas utilizando a métrica F1-Score Macro a fim de avaliar a qualidade entre os modelos.

\section{CONFIGURAÇÃO DO TREINAMENTO}

Todos os treinamentos ocorreram utilizando 8xV100 com o tamanho do lote de 128 (por GPU) e a cláusula de parada ocorre quando o loss do conjunto de dados de validação para de cair por um prazo de 10 épocas.

\section{EXPERIMENTOS}

Os resultados dos experimentos podem ser visualizados na \autoref{tab:experimentos}

\begin{table}[h]
\centering
\caption{Resultados dos Experimentos}
\label{tab:experimentos}
\begin{tabularx}{\textwidth}{r|c|c|c|c|c|c|c}
               & Departamento & Setor & Família & Subfamília & Média & Época & Horas \\
Texto + Imagem &              &       &         &            &  0.9  &   50  &   10h \\
Imagem         &              &       &         &            &       &       &       \\
Texto          &              &       &         &            &       &       &        
\end{tabularx}
\legend{Fonte: Elaborado pelo autor}
\end{table}
